
\section{项目概述}

我们将以复旦选课系统为例,完成一个简化版选课系统,其中体现数据库设计的思想。

\subsection{项目背景}
选课系统的实现分为客户端和服务器端两部分。而数据库设计的增删改查部分将在服务器部分体现。

\subsection{项目功能概述}

本项目的大致系统功能如下:(详细版如限制检查等请查看功能需求部分)

\begin{table}[h]
\begin{tabular}{ll}
{\bf 教师信息管理}   & 教师信息自动导入数据库 \\
{\bf 学生信息管理} & 学生信息自动导入数据库 \\
{\bf 课程开设}   & 课程信息的自动导入数据库 \\
{\bf 登陆/登出系统1}   & 学生/老师使用学/工号可登陆系统。 \\
{\bf 登陆/登出系统2}   & root管理员可增删改查所有信息。 \\
{\bf 学生选/退课}   & 学生可以查看目前课程的开设及选课情况,且在任意时间节点选/退课。 \\
{\bf 选课事务申请1}   & 学生对于选课人数已满的课程可以提交选课申请并查看申请状态。 \\
{\bf 选课事务申请2}   & 老师可以审核选课事务申请。 \\
{\bf 登分1}   & 管理员自动/手动录入学生课程成绩。 \\  
{\bf 登分2}   & 学生查看成绩。 \\ 
{\bf 登分3}   & 老师通过导入excel的方式自动登分。 \\ 
\end{tabular}
\end{table}


%插入泳道图 用况图

\subsubsection{教师信息管理}
系统管理员将教师信息(学院、工号、姓名等)手动或自动导入教师数据库中

\subsubsection{学生信息管理}
系统管理员将学生信息(学院、学号、姓名、专业等)手动或自动导入数据库

\subsubsection{课程开设}
课程信息(课程代码、课程名称、学分、任课老师、时间、地点、课时、人数/最大人数、考试时间等)的自动导入数据库

\subsubsection{登陆/登出系统}
学生/老师使用学/工号可登陆系统,查看其对应权限下的信息;root管理员可增删改查所有信息。

\subsubsection{学生选/退课}
学生可以查看目前课程的开设及选课情况,且在任意时间节点选/退课。

\subsubsection{选课事务申请}
学生对于选课人数已满的课程可以提交选课申请并查看申请状态;老师可以审核选课事务申请。

\subsubsection{登分}
管理员自动/手动录入学生课程成绩;学生查看成绩;老师通过导入excel的方式自动登分。




\subsection{项目模块划分}
 

\subsection{项目用户特征}
\subsubsection{教师}
教师可以使用工号登陆系统,查看教师权限下的所有信息(课程花名册/管理选课事务申请)。除此以外,教师还可以将对应课程的学生成绩导入系统。
\subsubsection{学生}
学生可以使用学号登陆系统,查看学生权限下的所有信息(成绩/课程表/选课申请/课程的开设及选课情况)。并且,学生可以对选课人数已达上限但仍没有超过教室容量的课程进行选课申请。
\subsubsection{管理员/院系}
院系(信息办工作人员)可以拿到root管理员权限,并进行数据的自动导入,可以查看所有信息并手动/自动增加条目,可以通过课程代码删除课程。


\subsection{开发框架}

\begin{table}[h]
\begin{tabular}{ll}
{\bf 开发语言}   & JavaScript / HTML / CSS / Python / SQLite \\
{\bf 浏览器环境} & Chrome / Safari \\
{\bf 第三方库}   & Ant-Design / Django                             
\end{tabular}
\end{table}






